\documentclass[10pt,a4paper]{article}
\usepackage[utf8]{inputenc}
\usepackage[T1]{fontenc}
\usepackage{amsmath}
\usepackage{amsfonts}
\usepackage{amssymb}
\usepackage[left=2cm,right=2cm,top=2cm,bottom=2cm]{geometry}
\author{Adam Urbaniak}
\title{Porównanie algorytmów planowania ścieżki dla robotów mobilnych}
\date{\today}

\begin{document}

\maketitle

\section{Omówienie problemu}
W dzisiejszych czasach roboty mobilne odgrywają coraz większą rolę w różnych dziedzinach życia, w tym w sprzątaniu i pielęgnacji terenów zielonych. Kluczowym wyzwaniem dla takich robotów jest efektywne planowanie ścieżki, aby mogły one poruszać się po swoim otoczeniu w sposób bezpieczny i efektywny.

Roboty mobilne, takie jak roboty sprzątające i koszące, są coraz powszechniej stosowane w różnych dziedzinach życia codziennego. Ich głównym zadaniem jest wykonywanie określonych czynności, takich jak sprzątanie domu czy koszenie trawników, w sposób autonomiczny i bez potrzeby ciągłego nadzoru człowieka. Jednym z kluczowych elementów skutecznego działania tych robotów jest zdolność do planowania optymalnej trasy poruszania się po otoczeniu.

Problem planowania ścieżki dla robotów mobilnych polega na znalezieniu najbardziej optymalnej trasy, która umożliwi robotowi dotarcie do celu w sposób bezpieczny i efektywny. W zależności od specyfiki zadania i warunków środowiskowych, mogą istnieć różne czynniki wpływające na proces planowania trasy, takie jak obecność przeszkód, zmienne warunki terenowe czy ograniczenia czasowe.

Algorytmy planowania ścieżki dla robotów mobilnych mają za zadanie rozwiązywać ten problem poprzez generowanie trajektorii, które umożliwią robotowi bezpieczne i skuteczne poruszanie się po otoczeniu. Istnieje wiele różnych podejść i technik do planowania ścieżki, które mogą być stosowane w zależności od specyfiki zadania i wymagań aplikacyjnych.

W kontekście robotów sprzątających i koszących, kluczowym wyzwaniem jest skuteczne omijanie przeszkód, takich jak meble w przypadku robotów sprzątających, lub zmienne warunki terenowe, takie jak nachylenia czy nierówności terenu, w przypadku robotów koszących. Dlatego też algorytmy planowania ścieżki dla tych robotów muszą uwzględniać specyficzne warunki środowiskowe oraz zapewniać efektywne omijanie przeszkód i adaptację do zmieniających się warunków.

\section{Porównanie algorytmów planowania ścieżki}
Teraz omówimy kilka algorytmów planowania ścieżki dla robotów mobilnych, ze szczególnym uwzględnieniem robotów sprzątających i koszących.

\subsection{Algorytm A*}
Algorytm A* jest jednym z najpopularniejszych algorytmów stosowanych do planowania ścieżki dla robotów mobilnych. Polega on na przeszukiwaniu grafu stanów, gdzie wierzchołki reprezentują kolejne stany robota, a krawędzie reprezentują możliwe ruchy. Algorytm A* wykorzystuje heurystykę, która szacuje koszt dotarcia do celu, co umożliwia efektywne znajdowanie optymalnej trasy.

\subsection{Algorytm RRT (Rapidly-exploring Random Tree)}
Algorytm RRT jest techniką probabilistyczną, która generuje drzewo stanów na podstawie losowych prób w przestrzeni konfiguracyjnej robota. Algorytm ten jest szczególnie skuteczny w przypadku robotów poruszających się w dynamicznych i nieprzewidywalnych środowiskach, ponieważ umożliwia szybkie dostosowanie trasy do zmieniających się warunków.

\subsection{Algorytm D* Lite}
Algorytm D* Lite jest modyfikacją algorytmu D*, która umożliwia efektywne planowanie ścieżki w dynamicznie zmieniających się środowiskach. Algorytm ten operuje na grafie stanów, aktualizując koszty dotarcia do poszczególnych wierzchołków w miarę odkrywania nowych informacji o otoczeniu. Dzięki temu, robot może szybko reagować na zmiany i adaptować swoją trasę do nowych warunków.

\subsection{Algorytm DFS (Depth-First Search)}
Algorytm DFS jest jednym z najprostszych algorytmów przeszukiwania grafu, który polega na eksplorowaniu środowiska poprzez głębokie przeszukiwanie. W kontekście planowania ścieżki dla robotów mobilnych, algorytm DFS może być stosowany do generowania trajektorii w przestrzeni konfiguracyjnej robota. Jednakże, ze względu na brak heurystyki, algorytm DFS może nie być najlepszym wyborem w przypadku robotów poruszających się w dużych i złożonych środowiskach.

\section{Porównanie wybranych algorytmów}
Porównajmy teraz wybrane algorytmy pod względem ich złożoności obliczeniowej, skuteczności w różnych warunkach środowiskowych oraz adaptacji do zmieniających się warunków.

\subsection{Złożoność obliczeniowa}
Złożoność obliczeniowa jest jednym z kluczowych czynników wpływających na wydajność algorytmów planowania ścieżki. Algorytmy, które mają niższą złożoność obliczeniową, mogą być bardziej efektywne w przypadku dużych i złożonych środowisk.

\subsection{Algorytm A*}
\begin{itemize}
    \item \textbf{Czas obliczeniowy:} W najlepszym przypadku, złożoność czasowa algorytmu A* jest rzędu \(O(1)\), ale w najgorszym przypadku może osiągać \(O(b^d)\), gdzie \(b\) to gałąź drzewa, a \(d\) to głębokość drzewa.
    \item \textbf{Przykład:} Dla prostego przypadku, gdzie droga jest prosta, liczba operacji może być bardzo mała, co skutkuje szybkim czasem wykonania.
\end{itemize}

\subsection{Algorytm RRT}
\begin{itemize}
    \item \textbf{Czas obliczeniowy:} Złożoność czasowa algorytmu RRT zależy od liczby iteracji potrzebnych do znalezienia optymalnej ścieżki. W zależności od skuteczności generowania losowych punktów, może to być rzędu \(O(n \log n)\) lub \(O(n^2)\), gdzie \(n\) to liczba wierzchołków w drzewie.
    \item \textbf{Przykład:} Dla prostej przestrzeni o niewielkiej liczbie przeszkód, algorytm może działać szybko i efektywnie.
\end{itemize}

\subsection{Skuteczność w różnych warunkach}
Niektóre algorytmy mogą być bardziej skuteczne w określonych warunkach środowiskowych, takich jak obecność przeszkód czy zmienne warunki terenowe. Dlatego też ważne jest, aby wybrać algorytm, który najlepiej odpowiada specyfice zadania i warunkom pracy robota.

\subsection{Adaptacja do zmieniających się warunków}
Roboty mobilne często muszą radzić sobie ze zmieniającymi się warunkami środowiskowymi, takimi jak pojawienie się nowych przeszkód czy zmiany w topografii terenu. Dlatego też istotne jest, aby wybrane algorytmy umożliwiały szybką adaptację do zmieniających się warunków i efektywne replanowanie trasy w razie potrzeby.

\section{Wnioski}
Analizując omawiane algorytmy planowania ścieżki dla robotów mobilnych, można zauważyć, że każdy z nich ma swoje zalety i ograniczenia. Wybór odpowiedniego algorytmu powinien zależeć od specyfiki zadania, warunków środowiskowych oraz wymagań aplikacyjnych. Istotne jest również ciągłe monitorowanie i rozwój technik planowania ścieżki, aby zapewnić skuteczne i efektywne poruszanie się robotów mobilnych w różnych warunkach.

\section{Referencje}
Dodatkowe informacje i źródła:
\begin{itemize}
    \item LaValle, S.M. (2006). Planning Algorithms. Cambridge University Press.
    \item Karaman, S., \& Frazzoli, E. (2011). Sampling-based algorithms for optimal motion planning. The International Journal of Robotics Research, 30(7), 846-894
    \item Hart, P.E., Nilsson, N.J., \& Raphael, B. (1968). A Formal Basis for the Heuristic Determination of Minimum Cost Paths. IEEE Transactions on Systems Science and Cybernetics, 4(2), 100-107  algorytm A*
\end{itemize}

\end{document}
